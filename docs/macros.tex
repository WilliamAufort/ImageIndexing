\documentclass[a4paper,10pt]{article}

\usepackage[utf8]{inputenc}
\usepackage[english]{babel}
\usepackage[T1]{fontenc}
\usepackage{mathpazo} %http://www.ctan.org/tex-archive/fonts/mathpazo
\usepackage{stmaryrd} %http://www.ctan.org/pkg/stmaryrd
\usepackage{amsmath} %http://www.ctan.org/pkg/amsmath
\usepackage{amssymb}
\usepackage{mathrsfs}
\usepackage{subfig}

\usepackage{cite}

\usepackage{amsthm} %http://www.ctan.org/pkg/amsthm
\usepackage{proof}
\usepackage{algorithm2e}

\usepackage[colorlinks=true]{hyperref} %http://www.ctan.org/tex-archive/macros/latex/contrib/hyperref/
\hypersetup{urlcolor=black,linkcolor=black}
\usepackage{footmisc} %http://www.ctan.org/tex-archive/macros/latex/contrib/footmisc

\usepackage{enumerate}
\usepackage{ulem} %http://www.ctan.org/tex-archive/macros/latex/contrib/ulem
\normalem
\usepackage{cancel} %http://www.ctan.org/tex-archive/macros/latex/contrib/cancel

\usepackage{fullpage} %http://www.ctan.org/tex-archive/macros/latex/contrib/preprint/
\setlength{\parindent}{0pt}
\setlength{\parskip}{\medskipamount}

\usepackage{pgffor}
\usepackage{tikz}
\usetikzlibrary{arrows,shapes.arrows, chains, positioning, automata, graphs, decorations.pathreplacing}

\usepackage{comment} %http://www.ctan.org/tex-archive/macros/latex/contrib/comment
\usepackage{multirow} %http://www.ctan.org/tex-archive/macros/latex/contrib/multirow
\usepackage{diagbox} %http://www.ctan.org/tex-archive/macros/latex/contrib/diagbox

\usepackage{textcomp} %http://www.ctan.org/pkg/textcomp

\usepackage{listings} %http://www.ctan.org/tex-archive/macros/latex/contrib/listings/
\lstset{numbers=left,language=Caml}

\usepackage{boiboites}
\newboxedtheorem[boxcolor=orange, background=blue!5, titlebackground=blue!20,
titleboxcolor = black]{theo}{Theorem}{compteurTh}
\newboxedtheorem[boxcolor=orange, background=blue!5, titlebackground=blue!20,
titleboxcolor = black]{defi}{Definition}{compteurDef}

\newcounter{ThComp}
\newcounter{DefComp}

\newboxedtheorem[boxcolor=red, background=red!5, titlebackground=red!50,titleboxcolor = black]{theoreme}{Theorem}{TheC}
\newboxedtheorem[boxcolor=orange, background=orange!5, titlebackground=orange!50,titleboxcolor = black]{definition}{Definition}{DefC}
\newboxedtheorem[boxcolor=blue, background=blue!5, titlebackground=blue!20,titleboxcolor = black]{proposition}{Proposition}{ProC}
\newboxedtheorem[boxcolor=cyan, background=cyan!5, titlebackground=cyan!20,titleboxcolor = black]{corollaire}{Corollary}{CorC}
\newboxedtheorem[boxcolor=black, background=black!0, titlebackground=black!20,titleboxcolor = black]{remarque}{Remark}{RemC}
\newboxedtheorem[boxcolor=green!70!black, background=green!70!black!5, titlebackground=green!70!black!30,titleboxcolor = black]{notation}{Notation}{NotC}
\newboxedtheorem[boxcolor=yellow, background=yellow!0, titlebackground=yellow!30,titleboxcolor = black]{exemple}{Example}{ExeC}
\newboxedtheorem[boxcolor=magenta, background=magenta!5, titlebackground=magenta!30,titleboxcolor = black]{lemme}{Lemma}{LemC}

\newcommand{\ra}{\rightarrow}
\newcommand{\la}{\leftarrow}


\newcommand{\RR}{\mathbb{R}}
\newcommand{\ZZ}{\mathbb{Z}}
\newcommand{\Ztwo}{\mathbb{Z}^{2}}

\newcommand{\ens}[1]{\left\{ #1 \right\}}
\newcommand{\set}[1]{\left\{ #1 \right\}}
\renewcommand{\leq}{\leqslant}
\renewcommand{\geq}{\geqslant}
\renewcommand{\le}{\leqslant}
\renewcommand{\ge}{\geqslant}
\newcommand{\cplx}[1]{\mathcal O \left( #1 \right)}
\newcommand{\floor}[1]{\left \lfloor #1 \right \rfloor}
\newcommand{\ceil}[1]{\left\lceil #1 \right\rceil}
\newcommand{\brackets}[1]{\left\llbracket #1 \right\rrbracket}
\newcommand{\donne}{\rightarrow}
\newcommand{\gives}{\rightarrow}
\newcommand{\dans}{\to}
\newcommand{\booleen}{\set{0,1}^*}
\newcommand{\eps}{\varepsilon}
\renewcommand{\implies}{~\Rightarrow~}
\newcommand{\tildarrow}{\rightsquigarrow}
\newcommand{\blank}{\texttt{\char32}}
\newcommand{\trans}[1]{\xrightarrow{#1}}
\newcommand{\rules}[1]{\xrightarrow{#1}}
\newcommand{\todo}[1]{\Large\textcolor{red}{#1}\normalsize}
\newcommand{\argmin}{\text{argmin}}
\newcommand{\rainbowdash}{\vdash}
\newcommand{\notrainbowdash}{\nvdash}
\newcommand{\rainbowDash}{\vDash}
\newcommand{\notrainbowDash}{\nvDash}
\newcommand{\Rainbowdash}{\Vdash}
\newcommand{\notRainbowdash}{\nVdash}
\newcommand{\bottom}{\bot}

